\documentclass{article}
\begin{document}

\section{minimax.c}


Program to calculate the min/max values of a data set.
Constructs a gmt cpt file, a shell script and executes it.
It automatically finds the number of columns in the data 
(2,3 or 4 are allowed). If 4 is used it is assumed that the data
structure is ID x y z. You can choose between 13 different color
tables (switch is). 

\subsection*{Command Line optinons}
\begin{tabular}[t]{lp{10cm}} 
-f $<Name>$ & Input file 2, 3 or 4D (default stdin)\\
-base $<Num>$ &  Percent of basemap larger than the data (Default =
2.0)\\
-prefix $<Name>$ & Prefix of all file names (.info, .cpt, .sh, .ps)
(default mm)\\
-info $<Name>$  & file name of the info file \\
-cpt $<Name>$   & file name of the cpt file \\
-gmt $<Name>$   & file name of the gmt script\\
-ps $<Name>$    & file name of the PS file \\
-prec $<Num>$   & precision of the cpt output (default 3)\\
-is $<Num>$     & switch for different colour scales (default =12)\\
-add            & switch to add a white segment between 0.0 and zmin
                   (default=0)\\
-pc $<Num>$     & percent added to dz to extend the colour scale 
                  (default = 0.0)\\
-nan $<Num>$    & ignore these values (default -99.0)\\
-notrun $<Num>$ & switch to undo the  excution of the script \\
-xsize $<Num>$  & Plot size x in inches (default 6.0)\\
-ysize $<Num>$  & Plot size y in inches (default xsize)\\
-title $<Name>$ & Plot title (default input file name)\\
-eps $<Num>$    & precision how good to hit nan\\
-checkz         & for 3 and 4 colums check if z=nan (default check
                  not)\\
-psxy           & display 3D data as points (no gridding)\\
-h$<Num>$       & Number of header lines in the input file. Works only
for 2D\\
-columns $<Num>$ & Forces the number of columns in the input file.
\end{tabular}
\end{document}